\documentclass[a4paper,12pt]{article}
\usepackage[a4paper, total={180mm, 272mm}]{geometry}

\usepackage{fontspec}
\setmainfont[Path=fonts/, Extension=.ttf]{ipaexm}

\setlength\parindent{3.5em}
\setlength\parskip{0em}
\renewcommand{\baselinestretch}{1.247}

\begin{document}

\thispagestyle{empty}

\Large
\noindent \\
Add Ino\medskip
\par
\normalsize
Pode ser encontrado na janela de FX Schematic em:\par
Add Fx->Layer\_Blending\_Ino->Add\_Ino\par
Adição simples de imagens.\par
Imagens ficarão mais claras usando adição.\par
Formula = Imagem de Backgorund + Imagem de Foreground.\\
\\
-{-}- \ Entradas \ -{-}-\par
Ambas as conexões são combinadas no processo.\par
Quando o interruptor da operação está desligado, o efeito mostra a imagem de Backgorund.\par
Ele faz o mesmo se existir apenas uma conexão.\\
Fore (Imagem de Foregrond)\par
Conecte a imagem que irá sobrepor a outra em cima.\\
Back (Imagem de Backgorund)\par
Conecte a imagem que ficará por baixo.\\
\\
-{-}- \ Configurações \ -{-}-\\
Opacidade (Opacity)\par
Especifica a opacidade da imagem que está sobrepondo.\par
Quando a opacidade está em 0 a imagem "Fore" será transparente.\par
O valor padrão é \textquotedbl 1.0\textquotedbl \ para a opacdade da imagem de Foreground e a soma das imagens é sintetizada como opaca.\par
Especifique um valor de 0 à 1,0.\par
É possível especificar valores maiores também (de 1 à 10.0).\\
\\
Máscara de Corte (Clipping Mask)\par
Quando a caixa de seleção estiver marcada,\par
Não existirá material nas áreas transparentes da imagem de Backgorund. Essas áreas ficarão vazias na imagem sintetizada.\par
O padrão para a máscará de corte é estar ligada.

O código para esse efeito pode ser encontrado em: opentoonz/toonz/sources/stdfx/ino\_blend\_add.cpp

\end{document}