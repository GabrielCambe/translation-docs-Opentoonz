\documentclass[a4paper,12pt]{article}
\usepackage[a4paper, total={180mm, 272mm}]{geometry}

\usepackage{fontspec}
\setmainfont[Path=fonts/, Extension=.ttf]{ipaexm}

\setlength\parindent{3.5em}
\setlength\parskip{0em}
\renewcommand{\baselinestretch}{1.247}

\begin{document}

\thispagestyle{empty}

\Large
\noindent \\
Negate Ino\medskip
\par
\normalsize
Invert each color channel of the image.\\
\par
\ \ 8bits Image: 0-{-}> \ \ \ 255、1-{-}> \ \ \, 254、... \ \ \ \ 254-{-}>1、 \ \ \, 255-{-}>0\par
16bits Image: 0-{-}>65535、1-{-}>65534、... 65534-{-}>1、65535-{-}>0\par
Values are inverted as in the above example.\\
\par
When there is an Alpha channel, it is inverted with respect to the Alpha values.\par
Alpha Value \ \ \ \ 0: 0-{-}> \ \ \, 0、1-{-}> \ \ \ 0、... 254-{-}>0、255-{-}>0\par
Alpha Value 128: 0-{-}>128、1-{-}>127、... 127-{-}>1、128-{-}>0、...255-{-}>0\par
Alpha Value 255: 0-{-}>255、1-{-}>254、... 254-{-}>1、255-{-}>0\par
Subsequently, the cell image will be reversed successfully.\\
\\
-{-}- \ Inputs \ -{-}-\\
Source\par
Connect the image to process.\\
\\
-{-}- \ Settings \ -{-}-\\
Specify whether to do the inversion for each color channel.\\
\\
Red \ \ \ \, Red channel inversion switch\\
Green \ Green channel inversion switch\\
Blue \ \ \, Blue channel inversion switch\\
Alpha \ \ Alpha channel inversion switch\\
\\
When OFF it does not do anything to that channel.\\
When ON, it will invert the target channel.\\
The default settings for Red, Green, Blue is ON, and Alpha is OFF.

\end{document}