\documentclass[a4paper,12pt]{article}
\usepackage[a4paper, total={180mm, 272mm}]{geometry}

\usepackage{fontspec}
\setmainfont[Path=fonts/, Extension=.ttf]{ipaexm}

\setlength\parindent{3.5em}
\setlength\parskip{0em}
\renewcommand{\baselinestretch}{1.247}

\begin{document}

\thispagestyle{empty}

\Large
\noindent \\
Motion Blur Ino\medskip
\par
\normalsize
Makes a blur of the camera by using a parallel shutter-style movement.\par
An afterimage shake effect is also possible to specify in the options.\\
\par
First, it will process the Alpha channel (if ON is specified), then,\par
it handles RGB pixels that have an Alpha channel not set to zero.\\
\par
The Alpha channel is not processed when (OFF) is specified,\par
to mask the change in the RGB image. Therefore, smooth edges will\par
remain smooth.\\
\\
-{-}- \ Inputs \ -{-}-\\
Source\par
Connect the image to process.\\
\\
-{-}- \ Settings \ -{-}-\\
Depend Move\par
\noindent \ \ \, P1 -> P2\par
The following X1,Y1,X2,Y2,\par
specify the move in a fixed direction and magnitude.\\
\par
\noindent \ \ \, Motion\par
In Geometry, it specifies the move E/W, from the H/S on a frame-by-frame basis.\par
The X1,Y1,X2,Y2 specified values will be ignored.\\
\\
X1\\
Y1\\
X2\\
Y2\par
Specifies the start and end point coordinate values of the parallel movement blur.\par
Coordinate system uses an origin at the lower left corner.\par
The unit is millimeters.\par
The following specified points, will give subtle changes in the length.\par
Distance between the start and end point will have no effect on 1/16 Pixel or more.\par
The default value is\par
\noindent \hskip 7em X1 Y1 -> 0.0 0.0\par
\noindent \hskip 7em X2 Y2 -> 1.0 1.0\par
for the points.

\newpage

\thispagestyle{empty}

\ \vspace{-0.2em}
\par
\noindent Scale\par
Specify the scale adjustment to the length of the parallel movement blur.\par
For example,\par
\noindent \hskip 7em X1 Y1 -> 0.0 0.0\par
\noindent \hskip 7em X2 Y2 -> 1.0 -1.0\par
\noindent \hskip 7em Scale -> 100\par
Will become,\par
\noindent \hskip 7em X1 Y1 -> 0.0 0.0\par
\noindent \hskip 7em X2 Y2 -> 100.0 -100.0\par
It will have the same effect.\par
If you specify a zero blur it will no longer have an effect.\par
The default value is 1 which will give no scale.\\
\\
Curve\par
Specify the adjustments to the strength of the blur.\par
With a value of 10.0 or less, and larger than 1, the blur becomes stronger,\par
with a value of 0.1 or more, and smaller than 1, the blur will become weaker.\par
The default value is 1 which will give equal attenuation.\\
\\
Zanzo Length\par
Specify the shift position of the afterimage effect\par
The unit is millimeters.\par
Specify a value greater than or equal to 0.\par
For example, when you want to shift the afterimage of a line with a width of 3,\par
specify a value greater than or equal to 3 for the afterimage.\par
The default value is 0 for no afterimage\\
\\
Zanzo Power\par
Determines the strength at the time of issuing the afterimage.\par
The weakest value is 0, where the afterimage effect will not be applied to the blur.\par
The larger the value,\par
the afterimage effect will become stronger, in the blur.\par
The default value is 1 the strongest. The blur will have the afterimage effect applied.\\
\\
Alpha Rendering\par
This is a valid switch only when there is an Alpha channel.\par
When OFF, it masks the changes in the RGB values using the Alpha value.\par
When ON, the process is also applied to the Alpha. There is no Mask.\par
The default setting is ON.

\end{document}