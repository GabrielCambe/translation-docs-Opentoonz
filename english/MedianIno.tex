\documentclass[a4paper,12pt]{article}
\usepackage[a4paper, total={180mm, 272mm}]{geometry}

\usepackage{fontspec}
\setmainfont[Path=fonts/, Extension=.ttf]{ipaexm}

\setlength\parindent{3.5em}
\setlength\parskip{0em}
\renewcommand{\baselinestretch}{1.247}

\begin{document}

\thispagestyle{empty}

\Large
\noindent \\
Median Ino\medskip
\par
\normalsize
Reduce noise, and erode majority of middle colors, rounds the contour of the picture\\
\\
-{-}- \ Inputs \ -{-}-\\
Source\par
Connect the image to process.\\
Reference\par
Connect the reference image to put the strength of the effect into each Pixel.\\
\\
-{-}- \ Settings \ -{-}-\\
Radius\par
Specify the range to be eroded by a circle radius.\par
The unit is mm.\\
\par
Specify a value greater than or equal to 0. The maximum is 100mm.\par
A value smaller than the pixel width (because you do not include the Pixel around)\par
will do nothing.\\
\par
The default value is 0.35mm.\\
\\
Channel\par
Specify the color channel to apply the median.\\
\par
\textquotedbl Red\textquotedbl\par
\textquotedbl Green\textquotedbl\par
\textquotedbl Blue\textquotedbl\par
\textquotedbl Alpha\textquotedbl\par
If you choose, to process over the specified color channel,\par
it will store the results in the RGBA channel.\par
In a black-and-white image, using this method, of single-channel processing,\par
the speed of processing will be faster.\\
\par
\textquotedbl All\textquotedbl\par
Using this, will multiply the processing to each RGBA channel.\\
\par
The default setting is \textquotedbl All\textquotedbl .\\
\\
Reference\par
Choose how Reference image values put the strength of the effect into each Pixel.\par
An image is connected to the \textquotedbl Reference\textquotedbl \ of the input,\par
Choose from Red/Green/Blue/Alpha/Luminance/Nothing.

\newpage

\thispagestyle{empty}

\ \vspace{-0.2em}
\par
Choose Nothing when you do not want this effect, it will turn off the connection.\par
The default setting is Red.

\end{document}